\documentclass[11pt]{article}

\usepackage[frenchb]{babel}
\usepackage[utf8]{inputenc}
\usepackage[T1]{fontenc}
\usepackage{graphicx}
\usepackage{syntax}
\usepackage{amsfonts}

\begin{document}
\title{Diagramme de décision binaire et Tetravex}
\author{Thibault Dardinier\\Gautier Izacard}
\maketitle
\tableofcontents
\newpage

\section{Formules logiques}
\subsection{Conversion d'une chaîne de caractère en formule logique}

Afin de convertir la chaîne de caractère (supposée correcte) reçue en entrée standard, nous procédons en deux étapes : analyse lexicale puis analyse syntaxique.

\subsubsection{Analyse lexicale}

Cette étape se décompose elle-même en deux sous-étapes.
La première, très simple, consiste en une simplification de la chaîne.
Les espaces sont supprimés, et tous les symboles (exceptés les entiers) sont réduits à un caractère : << <-> >> devient << = >>, << -> >> devient << > >>, << \&\& >> devient << \& >>, << || >> devient << | >>, << true >> devient << t >> et << false >> devient << f >>.

La seconde consiste à transformer la chaîne simplifiée en tokens (unités lexicales).
Pour cela, on lit récursivement les caractères de la chaîne, en gardant en mémoire le nombre courant (initialisé à $0$).
Si le caractère est un chiffre, on l'ajoute : $nombreCourant = 10 * nombreCourant + chiffre$.
Sinon dans le cas où le caractère n'est pas un chiffre, si le nombre courant est non nul, on l'ajoute au début de notre liste de tokens.
On convertit alors le caractère qui n'est pas un chiffre en token de l'opérateur qu'il représente.

\subsubsection{Analyse syntaxique}

Afin de convertir la liste de tokens en une formule, tout en respectant les priorités des différents opérateurs, on définit la grammaire suivante :

\setlength{\grammarparsep}{0.3em}
\begin{grammar}

<Équivalence> = <Implication> | <Implication> \Leftrightarrow{} <Équivalence>

<Implication> = <Disjonction> | <Disjonction> \Rightarrow{} <Implication>

<Disjonction> = <Conjonction> | <Conjonction> \lor{} <Disjonction>

<Conjonction> = <Brique> | <Brique> \land{} <Conjonction>

<Brique> = $n \in{} \mathbb{N}^{*}$
\alt \top
\alt \bot
\alt \neg <Brique>
\alt ( <Équivalence> )

\end{grammar}

Avec ces règles, lors d'un parcours de gauche à droite, il suffit simplement de regarder le prochain token pour savoir quelle règle doit être appliquée.
La conversion d'une chaîne de caractère en formule est donc faite en temps linéaire de la taille de l'entrée.

\subsection{Réduction du Tetravex à un problème de satisfiabilité}

Pour résoudre un Tetravex à partir de notre bibliothèque de manipulations de BDD, il faut réduire le problème à un problème de logique, c'est-à-dire convertir le problème en une formule, qui ne sera satisfiable que si le Tetravex est résoluble.
Pour cela, en notant $m$ le nombre de carrés, on considère $n.p.m$ variables, de type \textit{string} : {\NoAutoSpaceBeforeFDP{}"a,b:c"} avec $a \in{} [1, n], b \in{} [1, p], c \in{} [1, m]$, qui est vraie si, et seulement, le carré numéroté $c$ (à partir de 1) est à la position $(a, b)$ dans le Tetravex (qui commence en haut à gauche à $(1, 1)$).

À partir de là, on construit plusieurs formules, que l'on combinera ensuite.
La formule $impliqueDroite(a, b, c)$ dit que si on place le carré $c$ en $(a, b)$, alors il est nécessaire d'avoir (au moins) un carré compatible à droite (dont le chiffre à gauche est égal à celui de $c$ à droite).
Par exemple : $impliqueDroite(1, 1, 1) :=$ {\NoAutoSpaceBeforeFDP{}"2,1:2" \lor{} "2,1:3" \lor{} "2,1:5"}.
À noter que si un tel carré n'existe pas, la formule est simplement réduite à \bot{}.

On définit de la même manière $impliqueBas(a, b, c)$.

Un placement devant être unique (c'est-à-dire qu'un carré ne place au maximum qu'une seule fois), on définit $placerUnique(a, b, c)$ par la conjonction de la négation de tous les autres placements de $c$.
Par exemple, si $n = p = 2$ : $placerUnique(1, 1, 1) :=$ {\NoAutoSpaceBeforeFDP{} \neg "2,1:1" \land{} \neg "1,2:1" \land{} \neg "2,2:1"}.

À chaque emplacement, il doit y avoir (au maximum) un seul carré, d'où $aucunAutre(a, b, c)$.
Par exemple : $aucunAutre(1, 1, 1) :=$ {\NoAutoSpaceBeforeFDP{} \neg "1,1:2" \land{} \neg "1,1:3" \land{} \neg "1,1:4" \land{} \neg "1,1:5"}.

On les combine dans le cas d'un placement :

$placerPotentiellement(a, b, c) := ${\NoAutoSpaceBeforeFDP{}"a,b:c"}$ \Rightarrow{} impliqueDroite(a, b, c) \land{} impliqueBas(a, b, c) \land{} placerUnique(a, b, c) \land{} aucunAutre(a, b, c)$.

On combine alors tous les $placerPotentiellement(a, b, c)$ :

$toutPlacer := \land{} placerPotentiellement(a, b, c)$ $(\forall{} a, b, c)$

On a alors la garantie que si on a une solution, alors elle est valide.
Il manque donc simplement l'existence.

$exist(a, b) := {\NoAutoSpaceBeforeFDP{}"a,b:1" \lor{} "a,b:2" \lor{} "a,b:3" \lor{}}$...

$existence := \land{} exist(a, b)$ $(\forall{} a, b)$

$tetravex := existence \land{} toutPlacer$

$tetravex$ est la réduction du Tetravex à un problème de satisfiabilité. On peut donc le résoudre à l'aide des BDD.

\section{Génération des diagrammes de décision binaire et résolution du Tetravex}
\subsection{A partir d'arbres réduits}
Afin de construire le diagramme de décision binaire, nous avons dans un premier temps construit l'arbre de décision binaire puis nous l'avons réduit en suivant la démarche de l'énoncé.

On définit le type d'arbre binaire de décision comme étant un type somme de feuilles étiquettés par un booléen, et de noeuds ayant comme étiquette une variable de type abstrait, et deux fils qui sont des arbres binaires de décision, l'un situé à gauche, l'autre à droite. Pour construire l'abre on utilise une table de hashage pour mémoriser les valeurs booléenne que l'on a associées aux différentes varibles au cours du parcours du graphe : a chaque fois que l'on identifie une variable, on explore dans un premier temps le côté de l'arbre où cette variable prend la valeur faux puis l'autre côté où l'on fixe sa valeur à vrai. Lorsqu'il n'y a plus de variable dans l'expression, on évalue la formule booléenne avec la valuation ainsi défini.

Pour supprimer les noeuds qui n'interviennent pas dans l'évaluation de l'expression on parcourt l'arbre à partir des noeuds les plus bas en répérant les noeuds qui ont deux feuilles identiques, si un noeud a deux feuilles identiques remplace ce noeud par la feuille.

On définit ensuite le type d'un diagramme binaire : un diagramme binaire est définit par un identifiant de type entier que l'on veut unique, un objet du type abstrait Variable, un fils gauche qui est un diagramme de décision binaire, et un fils droit du même type.  

Pour passer au diagramme de décision binaire, on parcourt l'arbre afin de déterminer un identifiant pour chaque noeud. On utilise en parallèle une table de hashage, dont les valeurs des arbres de décisions binaires. Lorsqu'on traite un noeud, on calcule son identifiant, s'il est déjà dans la table on renvoie le noeud enregistré dans la table. Sinon on crée un nouveau noeud avec un nouvel identifiant.

Une manière naïve de procéder et de convertir l'arbre définit par le noeud courant en une formule propositionnelle, et de la voir comme une chaîne de caractères que l'on va utiliser comme clef.

Néanmoins cette approche n'est pas efficace. Une approche plus efficace consiste à assigner l'identifiant $0$ à la feuille $False$ et l'identifiant 1 à la feuille $True$. On définit ensuite la clef d'un noeud à partir des identifiants de ses fils et de la variable du noeud. Si la clef est présente dans la table, on retourne le noeud asssocié. Sinon on associe à ce noeud un identifiant qui n'a pas encore été utilisé. En pratique on utilise une référence que l'on incrémente à chaque nouveau noeud. 
\subsection{Par combinaisons de diagrammes}
On peut également construire le diagramme de décision binaire directement à partir de la formule propositionnelle. Pour se faire on définit les opérations logiques associés aux connecteurs sur les diagrammes de décision binaires. Par exemple on définit une opération $AndBDD$ qui prend en arguments deux diagrammes de décision binaires $Node(id1, v1, g1, d1)$ et $Node(id2, v2, g2, d2)$, et renvoie le diagramme obtenu en connectant les deux formules propositionnelles qu'ils représentent par l'opérateur logique $et$. Si les deux diagrammes ont le même identifiant, ce sont les mêmes et on renvoie l'un des deux diagrammes. Sinon on compare les deux variables $v1$ et $v2$ affectés à chacun des diagrammes.
\begin{itemize}
\item Si $v1 = v2$ on effectue l'opération binaire sur $g1$ et $g2$, ainsi que $d1$ et $d2$ avant de renvoyer le noeud créé à partir de la variable $v1$ et des deux diagrammes obtenus.
\item Si $v1 < v2$ on renvoie le noeud étiquetté la variable $v1$, à gauche on met le diagramme obtenu en effectuant l'opération sur les diagrammes $g1$ et $Node(id2, v2, g2, d2)$, à droite on a le diagramme obtenu en effectuant l'opération sur les diagrammes $d1$ et $Node(id2, v2, g2, d2)$.
\item Si $v1 > v2$ on effectue la même opération qu'au-dessus en échangeant les roles.
\end{itemize}
  
\subsection{Utilisation de la mémoïsation}
La complexité de l'opération binaire présentée à la section précédente est exponentielle en la taille de l'arbre, ce qui ne permet pas de résoudre le problème du Tetravex pour des dimensions supérieurs à $2$x$2$. Pour améliorer l'opération précédente on utilise un processus de mémoïsation : on utilise pour chaque opération binaire une table de hashage où les clefs sont de types $int * int$, déterminés par les identifiants des diagrammes binaires. Pour que l'identifiant soit unique la première composante de la clef est l'identifiant le plus petit. A chaque appelle de la fonction $AndBDD$ on vérifie que l'opération n'a pas encore été réalisée, sinon on effectue l'opération et on ajoute sa valeur dans la table. On fait de même avec l'opération unaire de négation en utilisant une table de hashage dont les clefs sont les identifiants des diagrammes de décision binaire. On définit les autres opérations à partir de ces trois opérations (on aurait également pu se limiter à 2 opérations). L'utilisation de ce procédé permet d'amortir le coût de l'opération. On peut ainsi résoudre des Tetravex dont les tailles vont jusqu'à
\end{document}

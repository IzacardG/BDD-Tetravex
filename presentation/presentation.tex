\documentclass{beamer}
\usepackage[utf8]{inputenc}              % Only when compiling
\usepackage[T1]{fontenc}                 % with PDFLaTeX.
\usepackage[frenchb]{babel}
\usepackage{algorithm}
\usepackage{algorithmic}
\usepackage{syntax}

\title{Diagramme de décision binaire et Tetravex}
\author{Thibault Dardinier \\ Gautier Izacard}

\begin{document}

\maketitle

\begin{frame}{Lecture d'une formule depuis une chaîne de caractères}
\begin{enumerate}
    \item Analyse lexicale
    \item Analyse syntaxique
\end{enumerate}
\end{frame}

\begin{frame}{Analyse lexicale}
    \begin{itemize}
            \item Suppression des espaces
            \item \<-> : =
            \item -> : >
            \item \&\& : \&
            \item || : |
            \item true : t
            \item false : f
    \end{itemize}

    \begin{enumerate}
        \item $(1 \Rightarrow{} 2) \Leftrightarrow{} (2 \lor{} \neg{}1)$
        \item \verb!(1 -> 2) <-> (2 || ~1)
        \item \verb![(, 1, ->, 2, ), <->, (, 2, ||, ~, 1, )
        \item Or (Imp (Var 1, Var 2), And (Var 2, Not (Var 1))) 
    \end{enumerate}
\end{frame}


\begin{frame}{Analyse syntaxique}

\setlength{\grammarparsep}{0.3em}
{\small
\begin{grammar}

<Équivalence> = <Implication> | <Implication> \Leftrightarrow{} <Équivalence>

<Implication> = <Disjonction> | <Disjonction> \Rightarrow{} <Implication>

<Disjonction> = <Conjonction> | <Conjonction> \lor{} <Disjonction>

<Conjonction> = <Brique> | <Brique> \land{} <Conjonction>

<Brique> = $n \in{} \mathbb{N}^{*}$
\alt \top
\alt \bot
\alt \neg <Brique>
\alt ( <Équivalence> )
\end{grammar}
}


\verb![(, 1, ->, 2, ), <->, (, 2, ||, ~, 1, )]

\end{frame}

\begin{frame}{Réduction du Tetravex à un problème de satisfiabilité}

{\small

    \begin{itemize}
        \item $impliqueDroite(1, 1, 1) :=$ {\NoAutoSpaceBeforeFDP{}"2,1:2" \lor{} "2,1:3" \lor{} "2,1:5"}.
        \item $impliqueBas(a, b, c)$
        \item $placerUnique(1, 1, 1) :=$ {\NoAutoSpaceBeforeFDP{} \neg "2,1:1" \land{} \neg "1,2:1" \land{} \neg "2,2:1"}.
        \item $aucunAutre(1, 1, 1) :=$ {\NoAutoSpaceBeforeFDP{} \neg "1,1:2" \land{} \neg "1,1:3" \land{} \neg "1,1:4" \land{} \neg "1,1:5"}.
        \item $placerPotentiellement(a, b, c) := ${\NoAutoSpaceBeforeFDP{}"a,b:c"}$ \Rightarrow{} impliqueDroite(a, b, c) \land{} impliqueBas(a, b, c) \land{} placerUnique(a, b, c) \land{} aucunAutre(a, b, c)$.
        \item $toutPlacer := \land{} placerPotentiellement(a, b, c)$ $(\forall{} a, b, c)$
        \item $exist(a, b) := {\NoAutoSpaceBeforeFDP{}"a,b:1" \lor{} "a,b:2" \lor{} "a,b:3" \lor{}}$...
        \item $existence := \land{} exist(a, b)$ $(\forall{} a, b)$
        \item $tetravex := existence \land{} toutPlacer$
    \end{itemize}
}

\end{frame}


\begin{frame}{Création du diagramme de décision binaire d'une formule propositionnelle}
Plusieurs pistes explorées :
\begin{itemize}
\item Création de l'arbre de décision binaire puis réduction vers le diagramme de décision binaire
\item Création du diagramme directement à partir de la formule propositionnelle
\end{itemize}
\end{frame}

\begin{frame}{Création de l'arbre de décision binaire}
Au cours du parcourt de la formule propositionnelle on construit la valuation progressivement en utilisant une table de hashage. 
\begin{itemize}
\item A chaque fois que l'on considère une variable qui n'est pas encore dans la table de hashage, on l'ajoute à la table.
\item On explore dans un premier temps le fils gauche en associant cette variable à faux.
\item Lorsqu'il n'y a plus de variables non affectées on évalue la formule avec la valuation ainsi construite.
\item Puis le fils droit en l'associant à vrai.
\end{itemize}
\end{frame}

\begin{frame}{Suppression des noeuds inutiles et passage au diagramme}
Suppressions des noeuds inutiles:
\begin{itemize}
\item Opération récursive
\item On commence par les noeuds les plus bas dans l'arbre
\item Si un noeud possède deux feuilles identiques, on le remplace par la feuille
\item Puis on remonte l'arbre
\end{itemize}
Pour passer au diagramme de décision binaire on souhaite déterminer un identifiant pour chaque noeud.
\begin{itemize}
\item On parcout l'arbre, si le noeud est reconnu, on le réutilise, sinon on le crée
\item Pour reconnaître le noeud on utilise une table de hashage dont les clefs sont les chaînes de caractères représentant la formule propositionnelle et les valeurs sont les noeuds.
\end{itemize}
\end{frame}

\begin{frame}{Identification plus efficace}
L'identification précédente n'est pas efficace car on ne stocke par sur chaque noeud la chaîne de caractères qu'il représente. Utiliser l'identifiant de chaque noeud permet d'être plus efficace à la fois au niveau de la table de hashage, et au niveau du parcourt de l'arbre.
\begin{itemize}
\item Déterminer la clef de chaque noeud dans la table de hashage en fonction des identifiants des deux fils et de la variable stockée par le noeud
\item Si la clef existe, on utilise le noeud déjà construit sinon on définit un identifiant qui n'est pas encore utilisé
\end{itemize}
\end{frame}

\begin{frame}{Création par combinaisons de diagrammes}
Création du diagramme directement à partir de la formule propositionnelle :
\begin{itemize}
\item Définition des opérations binaires et unaires définit sur les formules propositionnelles
\item Définition récursive des fonctions
\item Utilisation de l'identifiant pour ne pas combiner les mêmes noeuds
\item Utilisation de l'ordre sur les variables pour ordonner le diagramme
\item Sans mémoïsation lors de l'utilisation des opérateurs binaires cette approche n'est pas efficace
\end{itemize}
\end{frame}

\begin{frame}{Résultats sur le problème du Tetravex}
\begin{itemize}
\item Conversion de l'arbre vers le diagramme : 
\begin{itemize}
\item Tetravex $2$x$2$ résolu avec 4 carrés, mais débordement de pile à partir de 4 carrés. Utilisation du style par passage de continuation pour réduire la taille de la pile.
\end{itemize}
\item Création de l'arbre à partir de la formule propositionnelle, en utilisant les opérateurs binaires avec mémoïsation
\begin{itemize}
\item Résolution d'un tetravex $4$x$4$ en 5 secondes environ.
\end{itemize}
\end{itemize}
\end{frame}
	

\end{document}

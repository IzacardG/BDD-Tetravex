\documentclass[11pt]{article}

\usepackage[frenchb]{babel}
\usepackage[utf8]{inputenc}
\usepackage[T1]{fontenc}
\usepackage{graphicx}
\usepackage{syntax}
\usepackage{amsfonts}

\begin{document}
\title{Diagramme de décision binaire et Tetravex}
\author{Thibault Dardinier\\Gautier Izacard}
\maketitle
\tableofcontents
\newpage

\section{Formules logiques}
\subsection{Conversion d'une chaîne de caractère en formule logique}

Afin de convertir la chaîne de caractère (supposée correcte) reçue en entrée standard, nous procédons en deux étapes : analyse lexicale puis analyse syntaxique.

\subsubsection{Analyse lexicale}

Cette étape se décompose elle-même en deux sous-étapes.
La première, très simple, consiste en une simplification de la chaîne.
Les espaces sont supprimés, et tous les symboles (exceptés les entiers) sont réduits à un caractère : << <-> >> devient << = >>, << -> >> devient << > >>, << \&\& >> devient << \& >>, << || >> devient << | >>, << true >> devient << t >> et << false >> devient << f >>.

La seconde consiste à transformer la chaîne simplifiée en tokens (unités lexicales).
Pour cela, on lit récursivement les caractères de la chaîne, en gardant en mémoire le nombre courant (initialisé à $0$).
Si le caractère est un chiffre, on l'ajoute : $nombreCourant = 10 * nombreCourant + chiffre$.
Sinon dans le cas où le caractère n'est pas un chiffre, si le nombre courant est non nul, on l'ajoute au début de notre liste de tokens.
On convertit alors le caractère qui n'est pas un chiffre en token de l'opérateur qu'il représente.

\subsubsection{Analyse syntaxique}

Afin de convertir la liste de tokens en une formule, tout en respectant les priorités des différents opérateurs, on définit la grammaire suivante :

\setlength{\grammarparsep}{0.3em}
\begin{grammar}

<Équivalence> = <Implication> | <Implication> \Leftrightarrow{} <Équivalence>

<Implication> = <Disjonction> | <Disjonction> \Rightarrow{} <Implication>

<Disjonction> = <Conjonction> | <Conjonction> \lor{} <Disjonction>

<Conjonction> = <Brique> | <Brique> \land{} <Conjonction>

<Brique> = $n \in{} \mathbb{N}^{*}$
\alt \top
\alt \bot
\alt \neg <Brique>
\alt ( <Équivalence> )

\end{grammar}

Avec ces règles, lors d'un parcours de gauche à droite, il suffit simplement de regarder le prochain token pour savoir quelle règle doit être appliquée.
La conversion d'une chaîne de caractère en formule est donc faite en temps linéaire de la taille de l'entrée.

\subsection{Réduction du Tetravex à un problème de satisfiabilité}

\section{Génération des diagrammes de décision binaire et résolution du Tetravex}
\subsection{A partir d'arbres réduits}
\subsection{Par combinaisons de diagrammes}
\subsection{Utilisation de la mémoïsation}

\end{document}
